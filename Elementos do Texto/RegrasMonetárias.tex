\section*{Política Monetária}
Como dito no início, a política monetária é um instrumento da política econômica utilizado para estabilizar a economia. Ela impacta a economia afetando o custo do dinheiro, por meio do controle da taxa básica de juros, essa taxa pode afetar a inflação futura da economia, e outra execução de política econômica é o controle da quantidade de moeda em circulação na economia. Isso pode ser feito por uma alteração na taxa de recolhimento compulsório, operações com títulos públicos ou emitindo uma maior/menor quantidade de moeda. 	
\subsection*{Regra de Taylor}
O instrumento de política monetária principal que será analisado é a fixação da taxa básica de juros. Essa manipulação da taxa básica de juros é importante para manter a inflação em patamares baixos, o que é uma das metas universais de política econômica. A suposição de que o nível da taxa de juros teria impacto no nível da taxa de juros vem da Regra de Taylor. 

A Regra de Taylor é uma equação que relaciona a taxa básica de juros com o nível de inflação e o hiato do produto (diferença entre PIB potencial e PIB real) de um país. Com ela, em teoria, o banco central pode descobrir qual a taxa básica de juros necessária para chegar à meta desejada para a inflação em um período futuro. A equação é dada da seguinte forma:
\begin{equation}
    i - i^{*} = \alpha_{\pi}(\pi - \pi^{*}) + \alpha_{y}(Y - Y^{*})
\end{equation}
Em que:
\begin{itemize}
    \item $i$ é a taxa de juros real e $i^{*}$ é a taxa de juros real de equilíbrio
    \item $\alpha_{\pi}$ é o coeficiente de sensibilidade à variação da inflação
    \item $\pi$ é a taxa de inflação anual observada e $\pi^{*}$ é a meta de inflação
    \item $\alpha_{y}$ é o coeficiente de sensibilidade à variação do produto 
    \item $Y$ é o Produto Interno Bruto (PIB) e $Y^{*}$ é o PIB de pleno emprego dos fatores de produção. Podemos definir também $Y - Y^{*}$ como hiato do produto.
\end{itemize}
\subsection*{Regime de Metas para Inflação}
O Regime de Metas para Inflação é uma política econômica na qual o Banco Central de um país estabelece uma meta para o nível de inflação em um determinado período e se compromete a usar os instrumentos que tiver para alcançar essa meta. Os instrumentos principais utilizados são a manipulação da taxa de juros, da taxa de câmbio, expansão/redução da base monetária. 

O Banco Central anuncia uma meta, ao início de um período, se responsabiliza por atingir essa meta e torna a comunicação com o público mais transparente sobre o processo para cumprir a meta, buscando um retorno positivo das expectativas dos agentes econômicos. As expectativas dos agentes dentro de uma economia possuem um papel fundamental na eficiência de políticas econômicas.

No caso desse regime, a credibilidade possui um papel muito importante no desempenho do Banco Central para alcançar a meta. Se o BC se mostra comprometido em atingir a meta estabelecida, utilizando os instrumentos que possui de forma condizente com o objetivo, ele transmitirá maior credibilidade para a população, o que afeta positivamente as expectativas dos agentes da economia. Caso o BC mostre que não está usando os instrumentos de forma adequada para atingir a meta, não honrando com seu compromisso, ele perderá sua credibilidade com a população, o que afeta negativamente as expectativas dos agentes da economia e dificulta mais ainda o cumprimento da meta inflacionária. 

Como vimos anteriormente, a Regra de Taylor é uma regra amplamente utilizada por governos para controlar a inflação, por isso ela se relaciona ao Regime de Metas, que usa como instrumento o controle da taxa de juros. 
	
No caso do Brasil, o Regime de Metas usa como referência para o nível de inflação o IPCA (Índice de Preços do Consumidor Amplo). Por meio desse índice, o Banco Central avalia o funcionamento das suas políticas para atingir o nível de inflação desejado ao final do período. Segundo o próprio Banco Central brasileiro, o regime de metas brasileiro envolve: conhecimento público e prévio da meta, autonomia do banco central na adoção de medidas necessárias para atingir a meta, comunicação transparente e regular sobre os objetivos e justificativas das decisões da política monetária e mecanismos de incentivo e responsabilização para que o Banco Central cumpra a meta. 

O Banco Central brasileiro possui uma “taxa de tolerância” para essa meta, ou seja, existe uma faixa de valores para a inflação acima e abaixo da meta fixada que ainda é considerada aceitável e dentro da meta. Em 2021, a meta para a inflação do ano era de 3,75$\%$ a.a. com uma margem de tolerância de 1,5$\%$ para mais ou para menos, e a inflação foi de 10,06$\%$, estando, portanto, fora da meta. Quando casos como esse ocorrem, de a meta não ter sido cumprida, representantes do Banco Central devem ir a público prestar esclarecimentos sobre o porquê não foi possível atingir a meta estipulada.

