\section*{Introdução}
As políticas econômicas são ações realizadas pelo governo com o objetivo de alterar a situação econômica de um país, buscando atingir metas postas pelo próprio governo.  As metas de políticas econômicas consideradas universais são: manter os níveis de renda e emprego em patamares altos e os níveis de inflação em patamares baixos. Somando esses fatores, uma outra meta universal é o crescimento econômico.
Para atingir as metas citadas, existem instrumentos de políticas econômicas usados pelos governos dos países. Esses instrumentos são: a política fiscal, a monetária e a cambial. Nesse material, focaremos nas duas primeiras. 

De um modo geral, as políticas econômicas são importantes para manter um equilíbrio na economia, evitar crises duradouras, gerar crescimento econômico e garantir o bem-estar social para a população de um país. 

Nos módulos anteriores do curso, foram discutidos temas como Modelos de Oferta e Demanda Agregada, ciclo de negócios, Curva IS-LM. Saibam que todos esses assuntos são eixos coordenadores deste último módulo, o qual analisa separadamente a Política Monetária e Fiscal e seus efeitos macroeconômicos. 

Primeiramente, é necessário ressaltar que a política econômica passa por uma análise normativa que depende, principalmente, da visão de mundo do economista. Isso se mostra principalmente no embate se a política econômica deve ser ativa ou passiva. Liberais, defendem que a política econômica atue de modo passiva, dado que os preços são flexíveis, já em contrapartida os keynesianos defendem que ela atue de forma ativa, dado que os preços são rígidos. Já a análise positiva é feita a partir da IS-LM. 

Lembrando que análise positiva se refere, como mostra Marcio Garcia, professor do departamento de Economia da PUC-RJ: "a parte da ciência econômica que se preocupa com afirmativas capazes de serem verificadas pelos fatos" \cite{MarcioGarcia}. Isto é, as afirmações positivas deveriam ser redutíveis de forma a que ela possa ser testada por referência a evidência empírica. 

A economia normativa por sua vez, de acordo com o mesmo autor, se preocupa com juízos de valor, qual resultado tem impactos positivos ?, qual resultado tem impactos negativos ?, tais juízos são geralmente emitidos sobre os resultados a serem obtidos pelas políticas públicas. Assim, fazendo a ligação entre essas definições supracitadas e política econômica, Garcia conclui que:

\begin{citacao}
 "Os resultados concretos da política econômica, contudo, não dependem dos objetivos dos \emph{policy makers},objeto da economia normativa,mas dependem principalmente da cadeia de eventos que a ação tomada põe em movimento até determinar o resultado de
fato obtido, objeto da economia positiva." \cite{MarcioGarcia}
\end{citacao}

No fundo, a grande maioria dos debates sobre política econômica indica a ser sobre juízos de valor,aspecto normativo da ciência econômica. Porém, esses debates se centram, antes de tudo sob o aspecto postivo da economia, sendo assim portanto sobre o funcionamento da economia como um todo.
 
\subsection*{Política Econômica e Hiatos} 
Após essa definição prévia dos fatores que permeiam os \emph{policy makers}, um fator que pode ser tratado com mais cuidado é o da Incerteza, muitos dos modelos econométricos e macroeconômicos se baseiam em distribuições de probabilidade, quando conhecemos essa distribuição falamos de risco, quando não conhecemos falamos de incerteza, a postura ativa ou passiva da política econômica é geralmente permeado de incerteza. Incerteza, em si, significa  na prática que não sabemos a magnitude do efeito dessa política ou a sua duração, se ela ira gerar pontos positivos no futuro ou não, isso se chama efeito encadeamento. 

Para discorrer melhor sobre esse ponto, iremos abordar de forma simples, as duas principais contribuições de Robert Lucas para a macroeconomia, a teoria das expectativas racionais e a crítica de Lucas.

A Teoria das Expectativas Racionais, a grosso modo dizia que os agentes econômicos, se utilizando de toda a informação disponível,antecipam de forma racional o seu comportamento e assim reagem em consonância com as expectativas criadas e, de certa forma, anulam em algum grau a efetividade dessas políticas, gerando assim uma consistência interna em modelos que envolvem a incerteza. Quanto mais confiança os agentes econômicos tem sobre essa política econômica, mais rápido é o seu efeito, mais efetiva ela é. 

Já a Crítica de Lucas, se baseia no ponto em que os modelos macroeconômicos e econométricos, não poderiam ser usados para a formulação de política econômica, por mais que eles reconhecessem que as expectativas afetavam o comportamento, não as incorporavam de forma explícita, dado que os parâmetros são dinâmicos e não estáticos, e não se pode basear em parâmetros antigos,sempre em movimento, as melhores respostas estão sempre mudando. 

Com isso, o impacto de políticas econômicas pode ser definido como uma espécie de "faca de dois gumes",políticas econômicas internas afetam externas e vice e versa, afetando assim o produto, a inflação e o emprego. As reações a essas políticas são definidas por hiatos, sendo eles internos ou externos. 
\begin{itemize}
    \item Hiato Interno: Quanto tempo eu demoro para reagir a uma coisa que aconteceu ? 
    \item Hiato Externo: Agora que eu agi, quanto tempo leva para termos um efeito dessa ação na economia ?
\end{itemize}
Dentro do debate referente à política fiscal, o hiato interno é maior, dado o rito constitucional necessário para sua aprovação, como por exemplo as leis e regras orçamentárias (LDO, por exemplo). Já na política monetária o hiato interno é no máximo 45 dias, o intervalo de reuniões do COPOM. 

Em termos de hiato externo, a política monetária tem um efeito mais demorado, devido a questões envolvendo os seus efeitos e as decisões de consumo/investimento, enquanto na política fiscal o efeito é imediato, possuindo assim baixo hiato externo. 

Políticas econômicas possuem metas universais, elas são o nível do produto, o emprego e a baixa inflação, para se chegar a elas existe um sistema onde:
\begin{align}
    \begin{cases}
    \text{Target 1} (T_{1}) = a_{1}I_{1} + a_{2}I_{2} \\
    \text{Target 2} (T_{2}) = b_{1}I_{1} + b_{2}I_{2} \\
    \end{cases}
\end{align}
O modelo acima, denominado de Esquema de Tinbergen, determina que os instrumentos devem ser linearmente independentes, porém ai é que se encontra um conflito, a política fiscal e a política monetária não são independentes, existe um elo que interliga essas duas, a Taxa de Juros e sua relação com a variável de inflação ($\pi$) é estável. 

Sendo assim, reescrevemos a relação como: 
\begin{align}
    \begin{cases}
    q = a_{1}P_{m} + a_{2}P_{f} \\ 
    \pi = b_{1}P_{m} + b_{2}P_{f}
    \end{cases}
\end{align}
Revisando os tipos de políticas econômicas que podem acontecer:
\subsection*{Política Fiscal Expansionista e Política Monetária Contracionista}
    O aumento de gastos vai contra a contenção monetária,mais políticas fiscais monetariza levando a um aumento do crédito. Isso ocorreu no Brasil durante os anos 80 (Juros altos) e em 2013/2014 quando ocorria um aumento dos juros na parte monetária e no lado fiscal havia uma grande quantidade de subsídios. Esse tipo de política leva a um aumento da demanda agregada e do produto disponível na economia, porém também aumenta a inflação o que força um aumento na taxa de juros para controle. 
\subsection*{Política Fiscal Contracionista e Política Monetária Expansionista}
Uma política monetária expansionista indica um aumento na quantidade de moeda em circulação, sendo assim a autoridade monetária aumenta a taxa de juros para controlar a inflação, levando a uma diminuição do investimento, uma diminuição do consumo (Aumento da Taxa de Juros é um incentivo a poupança),queda do produto e uma queda na arrecadação de tributos e um aumento do gasto do governo com seguro desemprego e assistências de modo geral, piorando a situação fiscal em fluxo e em estoque, pois a dívida aumenta se a taxa de juros aumenta. 

Dessa forma, no fundo as políticas econômicas, na prática, são mais ativas que na teoria. Sendo assim, podemos nos questionar: Se as políticas econômicas são ativas, elas devem possuir regras ou ser livres ? Para responder essa pergunta, iremos pensar por partes: 
\begin{itemize}
    \item O que significa a política ter regras ? 
    
    Regras trazem compromisso, transparência, credibilidade, previsibilidade e consistência. Mostrando assim a possibilidade de reduzir o hiato interno, porém existem algumas críticas como: Regras feitas sob parâmetros não constantes podem não fazer sentido no futuro, pois elas devem ser revistas e possuir essa flexibilidade
    \item O que significa a política ser livre ? 
    
    Ser livre possibilita uma maior flexibilidade, levando a avaliação e a tomada de decisão para cada momento específico. Porém, existem algumas críticas: Poder enorme para as pessoas que não são qualificadas, podem ser inconsistentes no aspecto temporal e abrindo a brecha para uma maior interferência política. 
\end{itemize}

